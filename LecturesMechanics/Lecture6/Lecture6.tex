% !TeX program = pdflatex
% !TeX encoding = utf8
% !TeX spellcheck = uk_UA
% !BIB program = bibtex8

\documentclass[18pt]{LectMechanics}
\usetikzlibrary{patterns, snakes}

\tikzset{
	body/.pic = {
			\fill[red!40, draw=red, opacity=0.5]  plot[smooth cycle, tension=.7] coordinates {
			(-2.5,0.5)
			(-0.5,2.5)
			(1.5,2)
			(2,0.5)
			(1,-1.5)
			(-1.5,-1.5)
		};
	}
}



\usetikzlibrary{arrows.meta}

\title[Physics 1]{\huge\bfseries Non-inertial reference frame}
\date{}

\begin{document}
%=======================================================================================================
%\usebackgroundtemplate{
%
%\tikz\node[opacity=0.3]{\includegraphics[width=\paperwidth,height=\paperheight]{background}};%
%}
\begin{frame}
	\titlepage
\end{frame}
%=======================================================================================================
\usebackgroundtemplate{
}




%=======================================================================================================
\begin{frame}{Goals for Lecture}{}
	\begin{itemize}
		\item To uanderstand how to describe motion in non-inertial reference frame.
		\item To understand what inertial forces are and describe their types.
		\item To formulate the equation of motion in on-inertial reference frame.
	\end{itemize}
\end{frame}
%=======================================================================================================

%=======================================================================================================
\begin{frame}{Why is description of motion in non-inertial reference frame important?}{}
\begin{enumerate}
	\item Newton's second law $m\vec a =\vec F$ holds true only in inertial coordinate systems.
	\item However, there are many noninertial (that is, accelerating) frames that one needs to
	consider, such as elevators, merry-go-rounds, and so on.

	\item Is there any possible way to modify Newton’s laws so that they hold in noninertial
	frames, or do we have to give up entirely on $m\vec a =\vec F$? It turns out that we can in fact
	hold on to $m\vec a =\vec F$, provided that we introduce some new <<fictitious>> forces. These are
	forces that a person in the accelerating frame thinks exist.
	\item Consideration of noninertial systems will enable us to explore some of the
	conceptual difficulties of classical mechanics, and secondly it will provide deeper
	insight into Newton's laws, the properties of space, and the meaning of inertia.
\end{enumerate}
\end{frame}
%=======================================================================================================

%=======================================================================================================
\begin{frame}{}{}

\end{frame}
%=======================================================================================================

\end{document}