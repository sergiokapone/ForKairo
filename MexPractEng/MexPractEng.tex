% !TeX document-id = {1005c3c0-d1df-46ae-8c5c-0d3e509c55a8}
% !TeX program = lualatex
% !TeX encoding = utf8
% !TeX spellcheck = uk_UA
% !BIB program = biber

\documentclass[biblatex]{ProblemBookEng}
\usepackage{pdfrender}

%\usepackage{pscyr}
%========================================================================================================
%
\title{Problems}
\def\authors{S.~M.~Ponomarenko}
%
%========================================================================================================
%
\addbibresource{BooksEng.bib}

%========================================================================================================


%========================================================================================================
%
%									      Титульна сторінка
%
%========================================================================================================

\makeatletter
\newenvironment{alwayssingle}{%
	\thispagestyle{empty}
	\@restonecolfalse
	\if@twocolumn\@restonecoltrue\onecolumn
	\else\if@openright\cleardoublepage\else\clearpage\fi
	\fi}%
{\if@restonecol\twocolumn
	\else\newpage\thispagestyle{empty}\fi
}

\renewcommand\maketitle{
	\begin{alwayssingle}
		\begin{center} 	
			\large \MakeUppercase{NATIONAL TECHNICAL UNIVERSITY OF UKRAINE}\par
			\MakeUppercase{<<Igor Sikorsky Kyiv Polytechnic Institute>>}\par
			\MakeUppercase{Institute of Physics and Technology}\par
			%---Університет 
			\vspace*{5em}
			\large\sffamily\authors\par
			\vspace*{4em}
			{\par\centering
				\begin{tcolorbox}[width={\linewidth-6pt},
					breakable,
					enhanced,
					colback=white,
					colframe=white,
					arc=0pt, outer arc=0pt,
					boxrule=0pt,
					toprule=0pt,
					bottomrule=0pt,
					enlarge top by=5mm,
					]
					\centering
					\Huge\sffamily\bfseries\MakeUppercase{Physics 1. Mechanics}\\ 
					
					\vspace*{2em}
					
					\@title
				\end{tcolorbox}\par}
			
			\vfill
			
			\begin{center}
				For specialties
				
				\vspace*{1em}
				
				113 APPLIED MATHEMATICS
				
				125 CYBERSECURITY
			\end{center}
			
			%{\Huge\sffamily\bfseries\textcolor{black}{\@title}\par}	
			%{\vspace*{2em}\par\ifx\@logo\@empty\else\@logo\fi\par}
			%{\CYRF\kern-0.27ex\raisebox{-0.5ex}{\CYRII}\kern-0.27ex\raisebox{-0.1ex}{\CYRZ}\kern-0.27ex\TeX\par}
			%----Логотип 	
			{\vspace*{2em}}%---Відступ 
			{\large\sffamily\bfseries{}\par}
			\vfill
			%\ifelectronic \noindent Електронне видання \fi
			%\par {Версія від~\href{http://www.istpravda.com.ua/dates}{\today}} \par\else \par  \fi			       	
			{\normalsize\MakeUppercase{Kyiv~\the\year}\par}
			%{\normalsize\MakeUppercase{ВПІ ВПК <<ПОЛІТЕХНІКА>>}\par}
		\end{center}
		\newpage
	\end{alwayssingle}	
}

%========================================================================================================
%
%									      Друга сторінка
%
%========================================================================================================
\newcommand\makeinfopage{
	\begin{alwayssingle}
		
		Physics 1. Mechanics. \@title. For specialties 113 Applied Mathematics and 125~Cybersecurity / \authors{} -- Kyiv, <<Igor Sikorsky Kyiv Polytechnic Institute>>, 2018  -- ~\pageref{LastPage} p.
		
		
		\vspace*{6em}
		
		\begin{center}
			Electronic edition
		\end{center}
		
		Approved by methodical board of IPT Igor Sikorsky Kyiv Polytechnic Institute Protocol \textnumero~6 /2018 of May 24, 2018.
		
		\vspace*{5em}
		
		Author:
		\begin{itemize}
			\item Associate Professor of the Power System Physics Dept., Ph.D. Ponomarenko S.~M.
		\end{itemize}
		
		Reviewer:
		\begin{itemize}
			\item 	Associate Professor of the Information Security  Dept., Ph.D. Smirnov~S.~A.
		\end{itemize}
	
		\vfill
		
		In the editorship of the authors
		
		\vfill
		
		\begin{flushleft}\small
			 The typesetting of the text is done in the publishing system \LaTeXe{} (compiler Lua\LaTeX) on the basis of the computer typesetting system \TeX{} (Distributive  \href{https://www.tug.org/texlive/}{\TeX Live~\the\year}) using the editor \href{https://www.texstudio.org}{\TeX Studio}. The illustrative material of the book is prepared using the package \href{http://pgf.sourceforge.net}{TikZ/Pgf}.
		\end{flushleft}	
		\vfill
		
		\begin{flushleft}
			\begin{tabular}{p{\textwidth - 50ex}l}
				& \small\textcopyright\quad \authors, \the\year~р. \\
				& \small \textcopyright\quad IPT Igor Sikorsky Kyiv Polytechnic Institute, \the\year~р.           
			\end{tabular}
		\end{flushleft}
		\newpage
	\end{alwayssingle}
}
\makeatother

\begin{document}
\pagestyle{empty}
%\ifelectronic\CoverPage\setcounter{page}{0}\fi
\maketitle
\makeinfopage
%========================================================================================================
%
%									               Зміст
%
%========================================================================================================
\clearpage\pagestyle{main}
\tableofcontents
%========================================================================================================
%
%									         Вступ та передмова
%
%========================================================================================================
%% !TeX program = lualatex
% !TeX encoding = utf8
% !TeX spellcheck = uk_UA
% !TeX root =../EMProblems.tex

\chapter*{Передмова}

Дисципліна <<Електрика та магнетизм>> курсу <<Загальна фізика>>, який вивчається студентами Фізико-технічного інституту КПІ імені Ігоря Сікорського на другому курсі, входить до циклу базової підготовки студентів, що навчаються за спеціальністю 105 <<Прикладна фізика та наноматеріали>>. Велика частина програмного матеріалу, пов'язана з умінням розв'язування конкретних задач. Вироблення умінь, навичок і методів розв'язку величезного числа задач, звичайно, не може бути реалізована тільки за рахунок годин, відведених на практичні заняття, і має на увазі велику самостійну роботу студента.

Назви розділів та підрозділів посібника відповідають робочій програмі курсу. Практично до кожної задачі  в кінці збірника наведена відповідь, деякі задачі містять детальних розв'язок. Всі формули в основному тексті і відповідях наведені в гаусовій системі одиниць (СГС). Вихідні дані і числові відповіді надані з урахуванням точності значень відповідних величин і правил дій над наближеними числами. В кінці збірника дана таблиця основних фізичних констант та інші довідкові таблиці.

Дана версія посібника є електронним виданням, тому для зручності користування ним передбачена система навігації у вигляді гіперпосилань та бічної панелі змісту.

\vspace*{4em}

\begin{flushright}
	С.~М.~Пономаренко
\end{flushright}








%========================================================================================================
%
%									      Вставка файлів розділів
%
%========================================================================================================
\setcounter{chapter}{-1}
\newcommand{\Chapters}{%
	Vectors,
	Kinematics,
	Dynamics,
	NonInertialReferenceFrame,
	WorkEnergyConservationLaws,
	CentralForceProblem,
	RigidBody,
	SR,
	MechanicalOscillations,
}
%========================================================================================================
\multiinclude{\Chapters}[]
%========================================================================================================
%
%									       Вставка відповідей
%
%========================================================================================================
\answers
\multiinclude{\Chapters}[-Answers]
%========================================================================================================
%
%									             Додатки
%
%========================================================================================================
%\appendix
%\input{Additions/Vanaliz}
%\input{Additions/SItoGauss}
%\input{Spheres/Spheres}
%%\input{Additions/MatherialConstants}
%\input{Additions/PhysConstants}
%========================================================================================================
%
%									       Вставка бібліографії
%
%========================================================================================================
\newpage\appendixfalse\phantomsection\label{BibPage}

\nocite{%
	IrodovMechanics,
	BerkeleyMechanics,
	BerkeleyWaves,
	FLF1,
	Holyday,
	Crowell1,
	IrodovProblems}

\printbibheading
\defbibheading{subbibliography}[\bibname]{\section*{#1}}
\printbibliography[category=Textbooks, heading=subbibliography, title={Textbooks}]
\addtocategory{Textbooks}{%
	IrodovMechanics,
	BerkeleyMechanics,
	BerkeleyWaves,
	FLF1,
	Holyday,
	Crowell1,
}

\printbibliography[category=Problems, heading=subbibliography, title={Problem books}]
\addtocategory{Problems}{%
	IrodovProblems,
}


\end{document}
