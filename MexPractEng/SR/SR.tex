% !TeX program = lualatex
% !TeX encoding = utf8
% !TeX spellcheck = uk_UA
% !TeX root =../MexPractEng.tex

%=========================================================
\chapter{Theory of Special Relativity}\label{\currfilebase}
\Opensolutionfile{answer}[\currfilebase/\currfilebase-Answers]
\Writetofile{answer}{\protect\section*{\nameref*{\currfilebase}}}%
%=========================================================

\section{Kinematics of Special Relativity}

\subsection{Simultaneity and Time Dilation}

%=========================================================
\begin{problem}
	The mean lifetime of stationary muons is measured to be $2.2000$~\si{\micro\second}. The mean lifetime of high-speed muons in a burst of cosmic rays observed from Earth is measured to be $16.000$~\si{\micro\second}. To five significant figures, what is the speed parameter b of these cosmic- ray muons relative to Earth?
	\begin{solution}
		$0.990 50$.
	\end{solution}
\end{problem}

%=========================================================
\begin{problem}
	An unstable high-energy particle enters a detector and leaves a track of length $1.05$~mm before it decays. Its speed relative to the detector was $0.992c$. What is its proper lifetime? That is, how long would the particle have lasted before decay had it been at rest with respect to the detector?
	\begin{solution}
		$0.446$~\si{\pico\second}.
	\end{solution}
\end{problem}

\subsection{The Relativity of Length}


%=========================================================
\begin{problem}
	A rod lies parallel to the x axis of reference frame $K$, moving
	along this axis at a speed of $0.630c$. Its rest length is $1.70$~m.What
	will be its measured length in frame $K$?
	\begin{solution}
		$1.32$~m.
	\end{solution}
\end{problem}


%=========================================================
\begin{problem}
	A moving rod is observed to have a length of $2.00$~m and to be oriented at an angle of \ang{30.0} with respect to the direction of motion. The rod has a speed of $0.995c$. 
	\begin{enumerate*}[label=(\alph*)]
		\item What is the proper length of the rod?
		\item What is the orientation angle in the proper frame?
	\end{enumerate*}
	\begin{solution}
		\begin{enumerate*}[label=(\alph*)]
			\item $17.4$~m;
			\item \ang{3.30}.
		\end{enumerate*}
	\end{solution}
\end{problem}

\subsection{The Relativity of Velocities}

%=========================================================
\begin{problem}
	A spaceship whose rest length is $350$~m has a speed of $0.82c$ with respect to a certain reference frame. A micrometeorite, also with a speed of $0.82c$ in this frame, passes the spaceship on an antiparallel track. How long does it take this object to pass the ship as measured on the ship?
	\begin{solution}
		$1.2$~\si{\micro\second}.
	\end{solution}
\end{problem}


%=========================================================
\begin{problem}
	Galaxy A is reported to be receding from us with a speed of $0.35c$. Galaxy B, located in precisely the opposite direction, is also found to be receding from us at this same speed. What gives the recessional speed an observer on Galaxy A would find for 
	\begin{enumerate*}[label=(\alph*)]
		\item our galaxy and
		\item Galaxy B?
	\end{enumerate*}
	\begin{solution}
		\begin{enumerate*}[label=(\alph*)]
			\item $0.35c$; 
			\item $0.62c$.
		\end{enumerate*}
	\end{solution}
\end{problem}



\section{Dynamics of Special Relativity}

\subsection{Relativistic Energy and Momentum}


%=========================================================
\begin{problem}
	A proton ($m_p = 1.6726219 \cdot  10^{-27}$~kg) moves at $0.950c$. Calculate in eV ($1~\si{\eV} = 1.60217662 \cdot 10^{-19}\si{\joule}$) its
	\begin{enumerate*}[label=(\alph*)]
		\item rest energy,
		\item  total energy, and
		\item kinetic energy.
	\end{enumerate*}
	\begin{solution}
		\begin{enumerate*}[label=(\alph*)]
			\item $938$~MeV;
			\item $3.00$~GeV; 
			\item $2.07$~GeV.
		\end{enumerate*}
	\end{solution}
\end{problem}


%=========================================================
\begin{problem}
	What is the momentum in MeV/c of an electron ($m_e = 9.10938356 \cdot 10^{-31}$) with a kinetic energy of $2.00$ MeV?
	\begin{solution}
		$2.46$~ MeV/c.
	\end{solution}
\end{problem}


%=========================================================
\begin{problem}
	How much work must be done to increase the speed of an
	electron from rest to 
	\begin{enumerate*}[label=(\alph*)]
		\item $0.500c$,
		\item $0.990c$,
		and
		\item $0.9990c$?
	\end{enumerate*}
\end{problem}


%=========================================================
\begin{problem}
	An unstable particle at rest spontaneously breaks into two fragments of unequal mass. The mass of the first fragment is $2.50 \cdot 10^{-28}$~kg, and that of the other is $1.67 \cdot 10^{-27}$~kg. If the lighter fragment has a speed of $0.893c$ after the breakup, what is the speed of the heavier fragment?
	\begin{solution}
		$0.285c$.
	\end{solution}
\end{problem}


%=========================================================
\begin{problem}
	Particle A (with rest energy $200$~MeV) is at rest in a lab frame when it decays to particle B (rest energy $100$~MeV) and particle C (rest energy $50$~MeV). What are the
	\begin{enumerate*}[label=(\alph*)]
		\item total energy and
		\item  momentum of B and the
		\item total energy and
		\item momentum of C?
	\end{enumerate*}
	\begin{solution}
		\begin{enumerate*}[label=(\alph*)]
			\item $119$ MeV;
			\item $64.0$ MeV/c; 
			\item $81.3$ MeV; 
			\item $64.0$ MeV/c.
		\end{enumerate*}
	\end{solution}
\end{problem}

\subsection{Relativistic Equation of Motion}


%=========================================================
\begin{problem}
	A particle of rest mass $m$ starts moving at a moment $t = 0$ due to a constant force $F$. Find the time dependence of the particle's
	\begin{enumerate*}[label=(\alph*)]
		\item  velocity and
		\item of the distance covered.
	\end{enumerate*}
	\begin{solution}
		\begin{enumerate*}[label=(\alph*)]
			\item $v(t) = \frac{Fct}{\sqrt{m^2c^2 + F^2t^2}}$;
			\item $s(t) = \sqrt{\left( \frac{mc^2}{F}\right)^2 + c^2t^2  } - \frac{mc^2}{F}$.
		\end{enumerate*}
	\end{solution}
\end{problem}


%=========================================================
\begin{problem}
	A particle of rest mass $m$ moves along the $x$ axis of the frame $K$ in accordance with the law $x = \sqrt{a^2 + c^2t^2} $, where $a$ is a constant, $c$ is the velocity of light, and $t$ is time. Find the force acting on the particle in this reference frame.
	\begin{solution}
		$F = \frac{mc^2}{a}$.
	\end{solution}
\end{problem}


\Closesolutionfile{answer}

