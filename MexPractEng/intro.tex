% !TeX program = lualatex
% !TeX encoding = utf8
% !TeX spellcheck = uk_UA
% !TeX root =../EMProblems.tex

\chapter*{Передмова}

Дисципліна <<Електрика та магнетизм>> курсу <<Загальна фізика>>, який вивчається студентами Фізико-технічного інституту КПІ імені Ігоря Сікорського на другому курсі, входить до циклу базової підготовки студентів, що навчаються за спеціальністю 105 <<Прикладна фізика та наноматеріали>>. Велика частина програмного матеріалу, пов'язана з умінням розв'язування конкретних задач. Вироблення умінь, навичок і методів розв'язку величезного числа задач, звичайно, не може бути реалізована тільки за рахунок годин, відведених на практичні заняття, і має на увазі велику самостійну роботу студента.

Назви розділів та підрозділів посібника відповідають робочій програмі курсу. Практично до кожної задачі  в кінці збірника наведена відповідь, деякі задачі містять детальних розв'язок. Всі формули в основному тексті і відповідях наведені в гаусовій системі одиниць (СГС). Вихідні дані і числові відповіді надані з урахуванням точності значень відповідних величин і правил дій над наближеними числами. В кінці збірника дана таблиця основних фізичних констант та інші довідкові таблиці.

Дана версія посібника є електронним виданням, тому для зручності користування ним передбачена система навігації у вигляді гіперпосилань та бічної панелі змісту.

\vspace*{4em}

\begin{flushright}
	С.~М.~Пономаренко
\end{flushright}







