% !TeX program = lualatex
% !TeX encoding = utf8
% !TeX spellcheck = russian_english
% !BIB program = biber
\documentclass{rnp}
% ========================= Вихідні дані =================================
\def\rozrobnyk{Associate professor, Ponomarenko Sergiy Mykolajovych} % Розробник програми
\def\discipline{Physics 1. Mechanics} % Назва дисципліни
\def\cycle{fundamental natural science training}
\def\okr{bachelor} % освітньо-кваліфікаційний рівень
\def\methodcomday{24} % Число ухвалення метод момісією
\def\methodcommonth{May} % місяць ухвалення метод момісією
\def\methodcomnum{6/2018} % Номер протокола ухвалення метод момісією
\def\kafday{25} % Число ухвалення кафедрою
\def\kafmonth{April} % місяць ухвалення кафедрою
\def\kafnum{5/2018}% Номер протокола ухвалення кафедрою
\def\galuz{12 <<Information technologies>>}
\def\napriam{125 <<Cyber Security>>}
\def\spetsialnist{125 <<Cyber Security>>}
\def\spetsializatsija{}
\def\semestr{1}
\def\godyny{150}
\def\lectgodyny{36}
\def\practgodyny{36}
\def\labgodyny{18}
\pgfmathsetmacro\SRS{int(\godyny - \lectgodyny - \practgodyny - \labgodyny)}
\pgfmathsetmacro\kredits{\godyny/30}
\def\kontrol{exam}
% ==========================================================================
\addbibresource{EngMech.bib}

\def\nptitlepage{%
	\thispagestyle{empty}	
	\begin{center}
		\MakeUppercase{\large\Ministry}\\
		\MakeUppercase{\large\ntuu}\\
		\MakeUppercase{\large<<\kpi>>}\\
		%\ipt\\
	\end{center}
	
	\hfill\begin{minipage}{0.7\linewidth}
		APPROVED BY\\
		Deputy Director\\
		for scientific and pedagogical work\\
		\mfield{\linewidth}{}{}{IPT Igor Sikorsky Kyiv Polytechnic Institute}{(Institute/Faculty)}\\
		
		\mfield{0.2\linewidth}{}{}{}{(signature)} \mfield{0.75\linewidth}{}{}{T. V. Lytvynova}{(initials, surname)}\\ 
		
		\mfield{0.1\linewidth}{<<}{>>}{\methodcomday}{} \mfield{0.55\linewidth}{}{}{\methodcommonth}{}~\the\year~р.\\ 
	\end{minipage}
	
	\vfill
	\begin{center}
		\noindent\large\bfseries\discipline
	\end{center}	
	\vfill
	\begin{center}\large
		Program of the subject
	\end{center}
	for students preparing the first (Bachelor) level of higher education <<Bachelor>> of the specialties 113 Applied Mathematics and 125 Cybersecurity
	
	\vfill
	
	\hfill\begin{minipage}{0.7\linewidth}
		Approved by methodical commission\\
		\mfield{\linewidth}{}{}{IPT Igor Sikorsky Kyiv Polytechnic Institute}{(Faculty/Institute)}\\
		
		Protocol  №\mfield{2cm}{}{}{\methodcomnum}{}\mfield{0.8cm}{<<}{>>}{\methodcomday}{}\mfield{2.5cm}{}{}{\methodcommonth}{}~\the\year~р. \\
		Head of methodical commission:\\
		
		\inlinefield{2cm}{}{\phantom{text}}{(signature)} \inlinefield{0.83\linewidth}{}{S. A. Smirnov}{(initials, surname)}\\
		
		\mfield{0.1\linewidth}{<<}{>>}{\methodcomday}{} \mfield{0.6\linewidth}{}{}{\methodcommonth}{}~\the\year~р.\\ 
	\end{minipage}
	
	\vfill
	
	\begin{center}
		Kyiv -- \the\year
	\end{center}
	\newpage
	\thispagestyle{empty}
	
	Developer of program of the subject:
	
	associate professor, PhD, Candidate of Physico-Mathematical Sciences
	\newsavebox\mybox%
	\sbox{\mybox}{Ponomarenko Sergiy Mykolajovyсh}%
	
	\inlinefield{\wd\mybox}{}{Ponomarenko Sergiy Mykolajovysh}{} 
	
	
	\vfill
	The program is approved for the sessions
	
	of the Information Security department
	
	Protocol № \mfield{2cm}{}{}{\kafnum}{} of \mfield{1cm}{<<}{>>}{\kafday}{} \mfield{2cm}{}{}{\kafmonth}{} \the\year~year 
	
	Acting head of department, Associate Professor
	
	
	\inlinefield{2cm}{}{\phantom{text}}{(signature)} \inlinefield{6cm}{}{M. V. Graivoronsky}{(initials, surname)}\\
	
	
	and of the Mathematical Methods of Information Security department
	
	Protocol № \mfield{2cm}{}{}{4/2018}{} of \mfield{1cm}{<<}{>>}{17}{} \mfield{2cm}{}{}{\kafmonth}{} \the\year~year 
	
	Head of department, Full Professor
	
	
	\inlinefield{2cm}{}{\phantom{text}}{(signature)} \inlinefield{6cm}{}{M. M. Savchuk}{(initials, surname)}\\
	
	%\mfield{0.8cm}{<<}{>>}{\kafday}{}\mfield{2.5cm}{}{}{\kafmonth}{}~\the\year~р.
	
	\vfill
	\begin{flushright}
		\copyright~\kpi, \the\year~year 
	\end{flushright}
	\newpage
}
\begin{document}
\nptitlepage
%========================================================================================================
%
\section*{Introduction}
%
%========================================================================================================

Successful interaction of educational, organizational and methodological and scientific processes in higher educational institutions is the basis for the formation of a modern student, which provides the formation of a modern specialist with a full set of knowledge through the assimilation of all three components during the study period. Through this kind of work the student acquires skills of learning, assimilation, processing and use of new information.

Physics programs are based on one of the most advanced technologies -- the credit-module system, which ensures the improvement of the quality of education in the information society, the educational mobility of each student, the acquisition of self-realization and competitiveness in the labor market.

The discipline programs are aimed at maximizing the individualization of the learning process. The scientific and pedagogical staff of the department actively work on the new educational technologies of the educational process in accordance with the democratic values ​​of modern scientific and technological achievements of the Bologna process.

%The curriculum program <<\discipline>> is compiled in accordance with the educational-professional training program \okr{} training (specialties) \napriam{} (\spetsialnist).


The academic discipline belongs to the cycle of \cycle.


%========================================================================================================
%
\section{The purpose and tasks of the discipline}
%
%========================================================================================================

%========================================================================================================
%
\subsection{Purpose of the discipline}
%
%========================================================================================================

The basis of the credit-module technology of teaching disciplines \discipline{} are the following principles:
\begin{enumerate}
\item introduction of constant stimulation of independent learning by the student of educational material;
\item ensuring the regularity and continuity of training, increasing the value and objectivity of current and final control;
\item rejection of traditional forms of assessment of students knowledge, which strongly depend on the subjective approach of the teacher;
\end{enumerate}


In the study of academic disciplines, a comprehensive system approach to mastering the knowledge of the students, which allow timely adaptation to profound changes in Cyber Security technology, an increasing flow of information, and the latest scientific and technological advances in the field of information  technologies, is used.

%========================================================================================================
%
\section{Structure and main tasks of the discipline}
%
%========================================================================================================

The main forms of studying the discipline \discipline{} are lectures, practical and laboratory classes.

At lectures, students should familiarize themselves with the fundamental laws of nature, properties and structure of matter, to acquire physical theories, fundamental concepts and definitions of physical quantities, content of models, hypotheses, laws, principles, and to form a coherent modern physical picture of the world.

The material of the lecture by the teachers of the department is presented in the form of presentations using the technical means of study, which is provided by the department and the university as a whole. This form of lecture lecture provides an opportunity to provide students with broader and more in-depth information on the subject being studied, and also meets the modern requirements of innovative approaches in the methodology of teaching disciplines.

The practical lessons consolidate theoretical knowledge by solving problems and examples, assimilate the means and methods for solving specific problems from different sections of physics.

In laboratory classes, students get acquainted with the physical phenomena being investigated and laws to understand the essence of research methods, acquire the skills of assessing the technical means used in the experiment, the skills to establish the reliability of the results obtained, learn the ability to use statistical techniques and modern computer techniques for processing and analysis of the results of the experiment.

Individual work is done on teaching aids and textbooks. To facilitate students' independent work, the teacher defines the main and additional literature, provides methodological advice. Also, it is recommended for the individual work to use electronic versions of textbooks, manuals and textbooks, prepared at the department and presented in the university's library.

For the convenience of working out and raising the level of assimilation of material in the process of independent work, at the end of each section of each discipline, control questions and a list of recommended literature are submitted. The curricula for studying disciplines are aimed at increasing the level of mastering the lecture material, developing the scientific worldview, modern physical thinking and the skills of active independent educational, scientific, methodological and practical activities, forging the ability to put the experiment and conduct its analysis, develop the thinking and creative abilities necessary for the formation high level future specialist.


To study the discipline, \godyny~hours/\kredits ECTS credits are assigned.

%========================================================================================================
\begin{center}
\begin{tabular}{|c|c|c|P{1cm}||P{1cm}||P{1cm}||P{1cm}||P{1cm}||P{1cm}|c|}
\hline
\multirow{2}{*}[-1ex]{\rotatebox{90}{\parbox{3cm}{\centering Form of \\ study}}} 
& \multirow{2}{*}[-1ex]{\rotatebox{90}{\parbox{3cm}{\centering Semester}}}
& \multirow{2}{*}[-1ex]{\rotatebox{90}{\parbox{3cm}{\centering Total \\ credits/hours}}} 
& \multicolumn{6}{c|}{\makecell{Distribution of training \\ time by types of classes}}  
& \multirow{2}{*}[-1ex]{\rotatebox{90}{\parbox{3cm}{\centering Semester \\ certification}}} \\ \cline{4-9}
&                           
&                                        
& \rotatebox{90}{\parbox{3cm}{\centering Lectures}}
& \rotatebox{90}{\parbox{3cm}{\centering Practical \\training}}
& \rotatebox{90}{\parbox{3cm}{\centering Seminars}} 
& \rotatebox{90}{\parbox{3cm}{\centering Laboratory \\classes}}
& \rotatebox{90}{\parbox{3cm}{\centering Computer \\workshop}} 
& \rotatebox{90}{\parbox{3cm}{\centering Self-study}}
&                                       
\\ \hline
full-time                           
& \semestr                         
& \kredits/\godyny                       
& \lectgodyny     
& \practgodyny                
&                     
& \labgodyny                   
&                         
& \SRS  
& \kontrol                                 
\\ \hline
\end{tabular}
\end{center}
%========================================================================================================


%========================================================================================================
%
\section{Content of educational material}
%
%========================================================================================================

\begin{Rozdil}
\item Kinematics
	\begin{Rozdil}
	\item  Space and time. Reference frame. Material point. Radius vector. Trajectory, path, movement. Average and instantaneous velocity. Vector, coordinate methods for describing motion. Parametric description of the movement. Kinematics of the rotational motion. Rotational motion of a point and its angular characteristics.
	\item Kinematics of a rigid body. Progressive, rotational, flat motion of a solid body. Motion around a fixed point. Euler's theorem;
	\end{Rozdil}
\item Dynamics
	\begin{Rozdil}
	\item The concept of force. Types of interactions in modern physics. Types of forces. Newton's laws. Equation of particle motion, many-particle system. Momentum Conservation Law.
	\item Center of mass. Theorem on the motion of the center of mass. Reference frame of the center of mass. 
	\item Equation of motion of a body with a variable mass.
	\item Non-inertial reference frame. Inertial forces.
	\end{Rozdil}
\item Work and energy.
	\begin{Rozdil}
	\item Work of force. Power of force. Potential, conservative forces. Work of conservative forces. Potential energy and its normalization. Relationship between conservative force and potential energy. Gradient.
	\item Relationship between work of force and the change of the kinetic energy of a point. Kinetic energy of many-particle system. Kenig's theorem. Total mechanical energy conservation law. The work of dissipative forces.
	\end{Rozdil}
\item Angular Momentum Conservation Law.
	\begin{Rozdil}
	\item 	Angular momentum, torque (moment of force). Rate of change of angular momentum for particle and many-particle systems. Angular momentum conservation.
	\end{Rozdil}
\item Classical central-force problem.
	\begin{Rozdil}
	\item Central forces.
	\item Potential energy in central force field.
	\item Conservation Laws in central force field.
	\item Kepler's laws of planetary motion.
	\end{Rozdil}
\item Rigid body dynamics.
	\begin{Rozdil}
	\item The equation of motion of Rigid body.
	\item Relation between angular momentum and angular velocity.
	\item Rotation of the body relative to the fixed axis. Kinetic energy of rotating body.
	\item The moment of inertia of a particle, many-particle system and rigid body. Parallel axis theorem (Huygens–Steiner theorem).
	\end{Rozdil}
\item The principle of relativity
	\begin{Rozdil}
	\item Properties of space and time. Inertial reference systems. Galilean transformation and their effects. Postulates of the special theory of relativity. Properties of light speed in vacuum.
	\item Postulates of the special theory of relativity (SR). Properties of light speed in vacuum;
	\item  Lorentz transformations. The interval between events. World line Light Cone of Events. Types of intervals. Consequences of the Lorentz transformations~--- time dilation, relativity of simultaneity, the length contraction, velocity-addition formula.
	\end{Rozdil}
\item Dynamics or Special Relativity Theory
	\begin{Rozdil}
	\item Conservation of momentum and definition of relativistic momentum.
	\item Relativistic energy. 	Transformation of momentum and energy.
	\item Relativistic equation of motion.
	\end{Rozdil}
\item Mechanical oscillations.
	\begin{Rozdil}
	\item Oscillatory motion and its characteristics.
	Equation of oscillation. Harmonious oscillation. Graphic representation of harmonic oscillation by the method of vector diagrams.  Adding of one direction oscillations, perpendicular oscillations, beats.
	\item 	Representation of oscillations in a complex form. Differential equations of forced oscillations of an elastic pendulum, a mathematical and physical pendulum. Types of oscillations and their differential equations.
	\end{Rozdil}
\end{Rozdil}

%========================================================================================================
% 
\section{The approximative theme of practical training}
%
%========================================================================================================


The main purposes of the of practical training is to form the ability to use the knowledge of about  the laws of mechanics in professional activities.

\begin{enumerate}
\item  Kinematics of the particle.
\item  Kinematics of rotational motion, kinematics of a Rigid Body.
\item  Elements of Special Theory of Relativity.
\item  Particle dynamics and particle systems.
\item  Non-inertial frame or reference.
\item  Motion of a Rigid Body with a variable mass.
\item  Work and energy.
\item  Conservations Laws.
\item  Dynamics of a Rigid body.
\item  Motion in the field of Central Forces.
\item  Kinematics and Dynamics of the Special Theory of Relativity.
\item  Mechanical oscillations.
\end{enumerate}

%========================================================================================================
%
\section{The approximative list of laboratory classes}
%
%========================================================================================================


The main purposes of the laboratory classes are the formation of practical skills in the implementation of the experiment, processing and interpretation of its results.

During the semester, laboratory classes is carried out according to the schedule. For performing each laboratory work 3 hours~--- 2 for the experiment, 1 for work protection, individual work with the teacher. Topics of laboratory works:
\begin{enumerate}
\item Studying the motion of bodies in the field of gravity using the Atwood's machine.
\item Study of the laws of rotational motion on the example of the Oberbeck's pendulum.
\item Study of Conservation Lows of momentum and total mechanical energy on an example of collision of bullets.
\item Study of the motion of the physical pendulum.
\end{enumerate}


%========================================================================================================
%
\section{Individual calculation work} 
%
%========================================================================================================

In each semester, 1 calculation work is performed. It is executed as a semester problems and contains an individual set of problems of a calculated character, which covers the material of the semester. Variants of problems for the calculation work are given by the teacher at the beginning of the semester.

%========================================================================================================
%
\section{Control operations}
%
%========================================================================================================

1 Modular control work in semester.



%========================================================================================================
\begin{refsection}
	\section{Bybliography}
	\nocite{
	IrodovMechanics,
	IrodovProblems, 
	Holyday, 
	Crowell1, 
	BerkeleyMechanics,
	BerkeleyWaves,
	FLF1}
	\printbibliography[category=Main, title={Main}, heading=subbibliography]
	\addtocategory{Main}{
	IrodovMechanics, 
	IrodovProblems, 
	Holyday,
	BerkeleyMechanics,
	BerkeleyWaves}
	\printbibliography[category=Additional, title={Additional}, heading=subbibliography]
	\addtocategory{Additional}{%
	FLF1,  
	Crowell1}
\end{refsection}


%========================================================================================================
%
\section{The criterion for evaluating students' knowledge}
%
%========================================================================================================


The criterion for evaluating students' knowledge is the systemicity of general-professional knowledge, skills and abilities.

The control of students' current progress is carried out using the electronic journal, which records assessments for all types of student work, and this is the evaluation of the work in the classroom in practical classes, assessments for performing independent individual work, the protection of laboratory classes, estimates for the calculation and graphic work, modular control works. Together for the semester the student will receive more than 20 marks. Current performance is estimated at the 60-point system. At the end of each semester, the curriculum provides for a score or examination, the results of which are evaluated by the 40-point system.


\end{document}



